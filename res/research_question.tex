\documentclass[../paper.tex]{subfiles}

\begin{document}
  In the CampaNeo project, data is requested via specific campaigns, which must be
  approved by the data owner. Following the GDPR, campaigns must state exactly
  what the purpose of their data collection is and what type of processing they
  plan to do on it. Ideally, companies or research organisations behind the
  campaigns contribute to the development of better technologies and services
  in the realm of mobility and transport, which in turn enhance the user’s
  experience with the vehicle.

  For example, a campaign dedicated to enhance traffic flow around a city can collect
  GPS location and speed data from a big number of cars. This data can then help
  optimise traffic guidance and speed constraints on the roads, which leads to less congested roads and
  time savings for drivers.
  
  The Success of CampaNeo is highly dependent on requesting and receiving informed
  consent for sensor data sharing. The more people agree to send usage data from
  their cars, the more value the statistical analysis will generate. 
  
  We
  formulate two hypotheses which are the basic assumptions of this work:
  
  \begin{itemize}
    \item People are more willing to share their data if they are fully informed
          on what exactly they are sharing, when they are sharing it and with
          whom exactly they are sharing it.
    \item Data visualisations improve comprehension of consent.
  \end{itemize}

  It is debatable whether current consent gathering methods really make it
  absolutely clear to data subjects what happens in the background after they
  gave their consent. According to \cite{Borgesisus_informed_consent_2015}, people
  tend to agree to most consent requests they are confronted with.
  Reading through all the agreement specifications is time-consuming. Such
  documents are often written in a complex language typical for legal documents.
  Due to that, most people who give their consent to data sharing agreements do so
  without understanding many details of the contract. Bechman \cite{Bechmann2014}
  defines this as a “culture of blind consent”.
  In a large-scale field study of cookie consent notifications Utz et al. \cite{Utz_2019} experiment with different approaches of gathering user consent.
  
  One of these approaches tried to strictly apply GDPR's principle of data protection by default, meaning that the users have to actively check the boxes of partners they want to share their data with.
  They found that with this approach only 0.1\% of the users actively give their consent to the website which implies that most users are not fully aware of their choices.  
  A lack of awareness of personal data privacy was also found by Rader \cite{rader2014} who investigated a population sample on awareness of behavioural tracking on social media sites.
  
  To conclude, in most cases having
  one’s consent, even informed, is not equivalent to having awareness.
  In most cases, people are trying to get through the consent procedure as fast a possible and are not aware of the implications.
  To change this, we build an application to enable data owners to give informed
  consent and gain legal awareness in the process. This should be achieved by
  making the data flow completely transparent through a visualisation.

  %The research questions that are addressed in the design process are:

  %\begin{enumerate}
  %  \item What aspects of the data should be visualised?
  %  \item How can the data be visualised in order to improve comprehension?
  %\end{enumerate}
\end{document}
