\begin{document}
The CampaNeo project is highly dependent on requesting and receiving  informed consent for sensor data sharing. 
The more people agree to send usage data from their cars, the more value the statistical analysis will generate.
The underlying hypothesis of visualisation efforts for the CampaNeo project is  that people are more keen on sharing car sensor data if they are fully informed on what exactly they are sharing, when they are sharing it and with whom exactly they are sharing it.

It is debatable whether current consent gathering methods really make it absolutely clear to data subjects what happens in the background after they gave their consent. According to behavioural studies people tend to agree to most consent requests they are confronted with.\cite{Borgesisus_informed_consent_2015}
It takes too much time to read through all the specifications often also written in a complex language typical for legal documents.
This means that most people who give their consent to data sharing agreements do so without knowing many details of the contract at all. We must conclude that in the strict definition given by the GDPR, informed consent is not given in most of the cases where people agree to data sharing agreements.

To change this we build an application to enable data owners to give \textbf{informed consent} and gain \textbf{legal awareness} in the process. This should be achieved by making the dataflow completely transparent through a visualisation.

The questions that we need to address in the design process are what aspects of the data should be visualised and how can we visualise the data in order to improve comprehension of the process.

Once an appropriate prototype is implemented, the work should be targeted towards the following two research questions which can be verified during a testing phase with real users.

\begin{enumerate}
    \item Can we raise legal awareness of the data sharing process through transparent visualisation?
    \item Does the ability to constantly monitor one's data sharing activities make the data processor more trustworthy to data subjects?
\end{enumerate}


\end{document}