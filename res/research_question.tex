\documentclass[../paper.tex]{subfiles}

\begin{document}
  The CampaNeo project is highly dependent on requesting and receiving informed
  consent for sensor data sharing. The more people agree to send usage data from
  their cars, the more value the statistical analysis will generate. We
  formulate two hypotheses which are the basic assumptions of this work:

  \begin{itemize}
    \item People are more willing to share their data if they are fully informed
          on what exactly they are sharing, when they are sharing it and with
          whom exactly they are sharing it.
    \item Data visualisations improve comprehension of consent.
  \end{itemize}

  It is debatable whether current consent gathering methods really make it
  absolutely clear to data subjects what happens in the background after they
  gave their consent. According to \cite{Borgesisus_informed_consent_2015} people
  tend to agree to most consent requests they are confronted with.
  Reading through all the agreement specifications is time-consuming. Such
  documents are often written in a complex language typical for legal documents.
  Due to that, most people who give their consent to data sharing agreements do so
  without understanding many details of the contract. Bechman \cite{Bechmann2014}
  defines this as a “culture of blind consent”. To conclude, in most cases having
  one’s consent, even informed, is not equivalent to having awareness.

  To change this, we build an application to enable data owners to give informed
  consent and gain legal awareness in the process. This should be achieved by
  making the dataflow completely transparent through a visualisation.

  The research questions that are addressed in the design process are:

  \begin{enumerate}
    \item What aspects of the data should be visualised?
    \item How can the data be visualised in order to improve comprehension?
  \end{enumerate}
\end{document}
