\documentclass[../paper.tex]{subfiles}

\begin{document}
  This paper proposes an application and associated design approaches
  to visualise data sharing activities on the CampaNeo platform to
  the data owner. The tool visualises the different campaigns, which
  the users agreed to share their vehicle sensor data with. It is
  designed to present the user all the companies and institutions
  behind the data sharing campaigns and the detailed data retrievals
  in a time series visualisation.

  The work reasons about the different aspects of data, which is
  visualised and the features used to achieve a proper application
  and layout, like different technologies and methods.

  The conclusion was drawn from a user case study, which aims to
  highlight that the visualisation is helpful to improve the
  comprehension of data privacy rights. The results showed that 40\%
  of the testers, who did not believe that companies in the EU were
  respecting an individual's data privacy, changed their mind after
  the test with the visualisation tool. They were afterwards more
  convinced that user data of EU citizens can not be gathered without
  consent.

  The work shows that the availability of the presented tool can increase
  data sharing consent rates for CampaNeo campaigns by up to 35\%.
  Outside of CampaNeo, more than 30\% of the test users stated that
  they would be more likely to share different kinds of user data in
  services and applications they use, if being able to control or
  visualise the process with a dedicated application.

  We can therefore conclude that there is a clear need for better
  visualisation of data sharing streams. Furthermore, the case study
  shows that people feel more comfortable to share data, if they
  can easily oversee the exact activities in a visualisation tool.

  Since this is the first design iteration of this tool, the collected
  feedback ideas and comments will be implemented in future iterations
  to enhance the interface. Another test with a bigger sample group
  might then be scheduled for an improved prototype of the application
  as future work.

\end{document}
