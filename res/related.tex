\documentclass[../paper.tex]{subfiles}

\begin{document}
  The main problem we want to solve with our work is that of fully transparent
  visualisation of data sharing activities. We will achieve this by building a
  tool that enables users to monitor and control the distribution of their
  data.

  In recent years there have been several attempts to design applications that
  implement such visualisations. Raschke et al.
  \cite{core_privacy_dashboard_2018} built a general dashboard to visualise
  data sharing activities and give consent approval and withdrawal mechanisms.
  The dashboard is a single page application with a vertical timeline listing
  the different types of actions. Among these actions are sharing a first name
  or a picture as well as information about location and search history.
  Further, the application offers information about processing context and type
  of the data in question.

  The authors evaluated the tool with a set of tasks for participants to
  complete using the dashboard application. However, Raschke et al.
  \cite{core_privacy_dashboard_2018} only tested with expert users which were
  also their colleagues. The main takeaways were that data type categories
  need to be refined more to be understandable. Generally, even the expert users
  found it hard to answer questions about their data privacy based on the
  information available from the dashboard.

  Another implementation of a consent and data privacy visualisation interface
  is the Consent Request (CoRe) user interface (UI) from Drozd and Kirrane
  \cite{cure2020}.  The idea of the authors was to develop a
  UI which shows the implications of accepting a consent agreement. The
  hypothetical scenario would be consenting to the use of individual
  functionalities of a fitness tracker. For example, a user wants to have the
  route of a morning jog displayed on their app. To unlock the functionality, one
  has to accept some data processing by the data controller, which is the
  manufacturer of the tracker. The CoRe UI \cite{cure2020} will then display a
  graph that shows what data is sent out, where it will be stored, the type of
  processing that is done on it and which third party companies it will be
  shared with.

  To validate their design choices, Drozd and Kirrane evaluated
  two slightly different CoRe UI prototypes by giving tasks to participants
  from different age groups and recording their actions. The first prototype
  that was tested was a bit more elaborate and had more features than the
  second one which was a simplified version. For the first test 27~participants
  were asked to perform the specified tasks. 74\% of contestants
  seemed to be “very confused” and more than half of them found the UI “too
  complex” and “hard to use”. The simplified UI was accepted better with
  an even larger test group of 74~people. Still many described the layout as
  being “confusing”, “annoying” and “complex”. The majority of participants
  claimed to be unsatisfied or neutral with the application. The tolerance for
  cognitive overload through too much information and display of complex
  relations is low for most people. Especially when dealing with legal
  conditions of data privacy. Consequently, a good start for any attempt to
  create a transparent visualisation of data sharing processes is to simplify
  the user interface to only include the most essential components, which will
  be determined in \cref{sec:implementation}.
\end{document}
