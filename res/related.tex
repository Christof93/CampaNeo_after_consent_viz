\begin{document}
  As discussed before the problem we want to solve with our visualisation interface is that of fully transparent display of data sharing activities. We are going to build a tool that enables users to monitor and control the distribution of their information by the service they are using. 
  
  In recent years there have been several attempts to design applications that implement such visualisations. Raschke et al. \cite{core_privacy_dashboard_2018} built a general dashboard to visualise data sharing activities and give consent approval and withdrawal mechanisms. It is a very basic single page application with a vertical timeline listing the different types of actions. Among these actions we find examples like sharing a first name or a picture as well as information about location and search history. Additionally the application offers information about processing context and type of the data in question.
  
  The authors conducted a user experience analysis where they designed a set of tasks for contestants to perform using the dashboard application. Unfortunately they only tested with expert users among their colleagues. The evaluation concluded that the design was worth pursuing. The main take-aways were that data type categories need to be refined more to be understandable. Generally even the expert users found it hard to answer questions about their data privacy based on the information available from the dashboard.
  
  Another implementation of a consent and data privacy visualisation interface comes from Drozd and Kirrane \cite{cure2020}. Their consent request (CoRe) user interface (UI) seems to be even closer to the visualisation described in the introduction.
  The idea was to develop a user interface which shows all the implications of accepting a consent agreement. The hypothetical scenario would be consenting to the use of individual functionalities of a fitness tracker. For example a user wants to have the route of a morning jog displayed on his app. To unlock the functionality, one has to accept some data processing by the data controller which is the manufacturer of the tracker. The CoRe UI will then display a graph that shows what data is sent out, where it will be stored, the type of processing that is done to it and which third party companies it will be shared with.
  
  To validate their design choices, the authors evaluated two slightly different CoRe UI prototypes by giving tasks to participants from different age groups and recording their actions. The first prototype that was tested was a bit more elaborate and had more features than the second one which was a more simplified version. For the first test 27 participants were asked to perform the specified tasks. 74\% of contestants seemed to be very confused and more than half of them found the UI too complex and hard to use. The more simplified UI worked much better for an even larger test group of 74 people. Still many described the layout as being confusing, annoying and complex. The majority of contestants claimed to be unsatisfied or neutral with the application.
  It is important to note that the simplified prototype tested by Drozd and Kirrane is already quite minimal and should be easy to understand. The tolerance for cognitive overload through too much information and display of complex relations is low for most people. Especially when dealing with legal conditions of data privacy. Consequently, a good start for any attempt to create a transparent visualisation of data sharing processes is to simplify the user interface to the most essential components. 
\end{document}