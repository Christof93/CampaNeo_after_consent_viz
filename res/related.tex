\documentclass[../paper.tex]{subfiles}

\begin{document}

 \textit{Knowledge graph visualisation.} 
 The search for easily understandable visualisations of Semantic Web data is nearly as old as the Semantic Web itself.
 Many of the early visualisation efforts were components of Linked Open Data (LOD) browsers such as Isaviz \cite{isaviz}, RDF Gravity \cite{goyal2004rdfgravity} or Tabulator \cite{tabulator-berners-lee-2006}.  The LOD cloud is an interlinked network of openly available data sets.
 At the moment applications for the LOD cloud are mainly used by expert users such as ontology developers.
 A more widespread usage and adoption of Semantic Technologies would be very desirable for the usefulness of the Semantic Web itself.
 This is why making Semantic Web data more accessible to lay-users through comprehensive visualisations is a topic of ongoing research.
 One proposed way to achieve this are visualisation techniques which can present vast amounts of data while still getting across the semantic entities and relations.
 Stuhr et al. \cite{lodwheel} evaluate the suitability of several JavaScript visualisation frameworks for web-based data inspection tools.     
 
 Zhang et al. \cite{rdfFisheye} use a fish eye zoom feature to simplify information-dense visualisations.
 %comparable papers on knowledge graph visualisation.
 %Isaviz?, Open Link Data Explorer?, RDF Gravity?, Tabulator?
 \\\\
 \textit{Consent visualisation.}
  In recent years there have been several attempts to design applications that
  implement a transparent visualisation of data sharing mechanisms. Raschke et al.
  \cite{core_privacy_dashboard_2018} built a general dashboard to visualise
  data sharing activities and give consent approval and withdrawal functionality.
  The dashboard is a single page application with a vertical timeline listing
  the different types of actions. Among these actions are sharing a first name
  or a picture as well as information about location and search history.
  Further, the application offers information about processing context and type
  of the data in question.

  The authors evaluated the tool with a set of tasks for participants to
  complete using the dashboard application. However, Raschke et al.
  \cite{core_privacy_dashboard_2018} only tested with expert users most of which 
  were their colleagues. The main takeaway was that even the expert users
  found it hard to answer questions about their data privacy based on the
  information available from the dashboard.

  Another implementation of a consent and data privacy visualisation interface
  is the Consent Request (CoRe) user interface (UI) from Drozd and Kirrane
  \cite{cure2020}.  The idea of the authors was to develop a
  UI which shows the implications of accepting a consent agreement. The
  hypothetical scenario would be consenting to the use of individual
  functionalities of a fitness tracker. For example, a user wants to have the
  route of a morning jog displayed on their app. To unlock the functionality, one
  has to accept some data processing by the data controller, which is the
  manufacturer of the tracker. The CoRe UI \cite{cure2020} will then display a
  graph that shows what data is sent out, where it will be stored, the type of
  processing that is done on it and which third party companies it will be
  shared with.

  %To validate their design choices, Drozd and Kirrane evaluated
  %two slightly different CoRe UI prototypes by giving tasks to participants
  %from different age groups and recording their actions. The first prototype
  %that was tested was a bit more elaborate and had more features than the
  %second one which was a simplified version. For the first test 27~participants
  %were asked to perform the specified tasks. 
  %74\% of participants
  %seemed to be “very confused” and more than half of them found the UI “too
  %complex” and “hard to use”. The second, simplified UI was accepted better with
  %an even larger test group of 74~people.
  Upon testing the second, simplified version of the interface with 74 test users, they 
  found that despite some simplification, many would describe the layout 
  as being “confusing”, “annoying” and “complex”. The majority of participants
  claimed to be unsatisfied or neutral with the application.
  
  The tolerance for cognitive overload through too much information and display 
  of complex relations is low for most people. Especially when dealing with 
  legal conditions of data privacy. Consequently, a good start for any attempt
  to create a transparent visualisation of data sharing processes is to simplify
  the user interface to only include the most essential components, which will
  be determined in \cref{sec:prototyping}.
\end{document}
