\documentclass[../paper.tex]{subfiles}

\begin{document}
  % What are the results of the user tests?
  The testing period took place over the course of a week and comprised
  user tests with 17 participants. The age groups were quite evenly
  distributed between 16 and 50+ years, with the majority (53.9\%) of
  participants being in the age groups ranging from 16 to 26 years.
  Gender distribution was 70.6\% female, 29.4\% male. The education levels
  were leaning towards the higher end of the spectrum with 52.9\% having a
  high school level degree or equivalent and 41.1\% a university degree.
  All participants assessed themselves to be competent with internet
  surfing and most spend more than 4 hours a day on the internet. Also,
  76.6\% of the people are using social media on a daily basis.

  Since, the CampaNeo project is designed around cars, it was necessary to
  determine whether test subjects have a valid drivers license. Among
  the testers, 76.6\% confirmed to have one. Additionally, 52.9\% stated
  that they even own a car themselves. 47.1\% of the participants drive daily
  and use the car as their main mean of transport.

  In the testing phase we noticed that the test users had no problems solving
  the tasks, that were given to them. There were only a few exceptions, where
  tasks couldn't be solved without help. The users understood the interface by
  mostly relying on their intuitions and recognition of symbols.

  After the practical tasks, the participants rated the design of the
  application with a mean score of 3.3 on a scale from 1 to 4. This shows
  that our first prototype was received well overall. Additionally, the
  application can be improved with the insights from the test.

  The users were also asked to choose from a set of adjectives, which they
  found to be describing the interaction with the application best. Among the
  most chosen ones were ``organised'', ``innovative'' (both 70.6\%) and
  ``effective'' (47.1\%). The more negatively associated adjectives like
  ``complex'', ``hard to use'' and ``useless'' were only selected by one person
  each, which shows a very positive result overall. This is further strengthened
  by the fact that 80\% of the participants stated they were ''Very satisfied``
  with the user interface.

  When looking at the rated understandability, we see that 88.3\% of the users
  claimed to ``see who gets my data'' using the application. 76.5\% also said
  that the application improved their understanding of what happens to their
  personal data, showing a clear increase in the users awareness of the data
  sharing process. However, just 47.1\% agreed that they became ``more confident
  in their knowledge of data sharing'', which is probably related to many users
  being already aware of the most basic GDPR rights and did not gain a lot of
  knowledge through the application alone. This could be the result of the high
  presence of GPDR regulations in the media, when they were introduced.

  To further assess the participants' awareness of topics surrounding data
  privacy, we designed a set of questions, which were tested before and after
  the usage of the visualisation tool. It was asked if the participants trust
  companies in the EU to respect their data privacy, which the majority answered
  with ``yes'' or ``probably'' (combined 64.7\%).
  The remaining 6 people (35.3\%) were more sceptical about GDPR adherence of companies or were not even aware of the GDPR.
  2 of these 6 changed their mind on this particular question after the interaction with our visualisation tool.
  This indicates that they got more aware of the regulations companies must follow in the EU and felt more secure after the insights of the visualisation.

  We wanted to know how the participants react when they are asked to share their data for research and development purposes in services and applications in their daily life.
  58.8\% said that they sometimes or even always (5.9\%) share data when they come into such a situation.
  Recalling the reasons why they gave their consent, only 1 person admitted to read data privacy agreements and explicitly agreeing with the contract.
  41.2\% said they wanted to contribute to improve the user experience, but many also indicated that they ``felt like they had to agree to use the service'' (35.3\%), or even gave their consent without realising (11.8\%).
  After finishing the test we asked if the participants would share data if they had the tested tool at their disposal for any kind of service or application that needs user data.
  % increase of 20-30%
  23.5\% changed their mind from sharing data only rarely to sharing it possibly and 35.3\% indicated that they were more likely to share data when they had the tool than without it.
  The same question phrased more specifically towards CampaNeo and sharing vehicle data like GPS location, speedometer data or fuel gauge readings gives an even clearer picture.
  % increase of 40%
  Before the test, 58.8\% said they'd share at least selected types of data with campaigns on CampaNeo. After the interaction with the tool, 94.1\% of the participants said that they would possibly share their vehicle data if they had the tool to control the sharing activities.
\end{document}
