\begin{document}
  Since it has come into effect in 2018, the general data protection regulation (GDPR) has had a big impact on the way companies deal with user data of all kinds.
  The GDPR brought awareness of data ownership and protection of privacy to a new level for companies as well as consumers.
  There is a big interest to handle data generated by activities of real people in a safe and consensual way. For many companies adhering to the regulations is not only important to prevent costly legal affairs but also to keep up a good reputation with their customers.
  At the same time, customers are more and more concerned about their rights as data owners and individuals. Many value their privacy high and want to control their digital footprint own their own.
  
  The GDPR requires \textbf{consent} between a data owner and a data controller if the data controller wants to process any kind of information related to the data owner.(Art. 6, 1a) This consent has to be freely given, specific, informed and unambiguous. (rec 32)
  
  The focus of the work presented in this paper is on establishing truly informed consent, meaning that a data owner is completely aware of extent, target and content of his data sharing activities. 
  To achieve this we look into general ways to improve transparency with the data owner through informative visualisations. We believe that clever visualisations help users to get an overview of their data sharing situation more quickly and less mentally tiresome than through endless paragraphs of text which is the prevalent status quo.
  
  When it comes to data visualisation for an end-user, the most important aspect is the simplification of complex data. Data for applications is generally stored in some sort of database system, opposing a more complex schema, which is very complicated and hard to understand for non-experts unfamiliar with the domain. Therefore, a step of transformation has to be used to represent selected data to a user in a readable manner. Visualisation is a technique of mapping such complex data to graphical elements, thus making it easier to understand relations within the dataset.

  This paper presents an approach to visualise data in the domain of the \textit{CampaNeo} project. CampaNeo is a joint venture project between several companies in the automotive area and academia. The goal is to create a system to collect and distribute sensor data generated by modern vehicles.
  Requests fro data can be issued by so called campaigns which must be approved by the owner or user of the vehicle. Campaigns must state exactly the use-case of their data collection and the processing they are planning to do on it. Ideally companies or research organisations behind the campaigns contribute to the development of better technologies and services in the realm of mobility and transport which in turn enhance the user's experience with the vehicle.
  
  For example, location and speed data from a big number of cars can help to optimise traffic flow management which leads to less congested roads and time savings for drivers.
  
  With the many benefits of semantic knowledge graphs for building versatile, explainable and machine-readable data storage solutions they are the underlying technology used in the CampaNeo project. 
  Since semantic data from a \textit{triplestore} mainly describes objects and their relations, it can be quite complex to visualise. One could just display the whole database as a graph, with all objects and their relations, however, this would result in a very large visualisation which is just too cluttered and therefore confusing.
  % TODO: cite CampaNeo
   The idea is to visualise the flow of data from a user's car to third party companies on a small to medium display, typically found on larger smartphones or tablets. The user can then get an overview on what data they are sharing with institutions like governmental agencies, universities or data processing companies who collect high amounts of data with the intent to solve problems around mobility and transport. The solutions generated from the data should in turn benefit the data owner. The visualisation focuses on highlighting the data streams to the user who should get information about the type of data that is shared, at what intervals it is sent out and who the receiving party is.
  
  In the following section we show two state of the art approaches for similar visualisation interfaces. We will point at their shortcomings and define our research question to in order to find the visualisation that solves our specific problem. The fifth section presents the initial prototype and the mock-ups. Then, we describe the architecture we used to build our first implementation. In the evaluation section, we explain how we tested our research question with a user-study and present the results of it. The last section presents a small conclusion and an outlook to future work.
\end{document}