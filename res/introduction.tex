\documentclass[../paper.tex]{subfiles}

\begin{document}
  Since it has come into effect in May 2018, the General Data Protection
  Regulation (GDPR) has had a big impact on the way companies deal with
  personal data. The GDPR brought awareness of data ownership and protection
  of privacy to a new level for both companies and consumers. There is an
  interest to handle data generated by activities of real people in a safe and
  consensual way. For many companies adhering to the regulations is not only
  important to prevent costly legal affairs but also to keep up a good
  reputation with their customers. At the same time, data owners are more and
  more concerned about their rights. Many value their privacy highly and want
  to control their digital footprint on their own.

  The GDPR requires consent between a data owner and a data controller
  if the data controller wants to process any kind of information related to
  the data owner (Art.~6~(1a)). According to the GDPR, consent has to be
  (i)~freely given, (ii)~specific, (iii)~informed and (iv)~unambiguous (Rec.~32).

  This paper focuses on establishing informed consent; that means a data owner
  is completely aware of extent, target and content of his data sharing
  activities. To achieve this we look into ways to improve transparency with
  the data owner through informative visualisations. We believe that
  visualisations help end users understand how their data is being shared,
  with less effort than by reading the agreement text which is the prevalent
  status quo.

  When it comes to data visualisation to end users, the most important aspect
  is the simplification of complex data. %TODO: ref
  Data for applications is generally stored in some sort of database system
  which is hard to understand for non-expert users. Data visualisation is a
  technique of mapping complex data to visual elements, thus making it easier
  to understand relations within the dataset.

  This paper presents an approach to visualise data in the domain of vehicle
  sensor data sharing. The work is part of the CampaNeo project, whose goal
  it is to create a system to collect and distribute sensor data generated by
  modern vehicles. Data is requested via specific campaigns, which must be
  approved by the data owner. Following the GDPR, campaigns must state exactly
  what the purpose of their data collection is and what type of processing they
  plan to do on it. Ideally, companies or research organisations behind the
  campaigns contribute to the development of better technologies and services
  in the realm of mobility and transport, which in turn enhance the user’s
  experience with the vehicle.

  For example, GPS location and speed data from a big number of cars can help
  optimise traffic flow management, which leads to less congested roads and
  time savings for drivers.

  Semantic knowledge graphs are a state-of-the-art solution for building
  versatile, explainable and machine-readable data storage solutions. Knowledge
  graphs are the underlying technology used in the CampaNeo project. Since
  semantic data from a \textit{triplestore} mainly describes objects and their
  relations, it is complex to visualise. One could just display the whole
  database as a graph, with all objects and their relations, however, this
  would result in a very large visualisation which is too cluttered and
  therefore confusing to users. % TODO: cite CampaNeo

  The idea is to visualise the flow of data from a user’s car to third party
  companies on small to medium displays (e.g. tablet, smartphone or the car’s
  built-in infotainment system). The user can get an overview on what data they
  are sharing with institutions like governmental agencies, universities or
  data processing companies who collect high amounts of data with the intent to
  solve problems around mobility and transport. The solutions generated from
  the data should in turn benefit the data owner in some way. The visualisation
  focuses on highlighting the data streams to the user who should get
  information about the type of data that is shared, at what intervals it is
  sent out and who the receiving party is.

  This paper is structured as follows: \Cref{sec:introduction} presents an
  introduction to the field, while \cref{sec:related_work} presents related
  work. \Cref{sec:research_question} defines the main research questions. The
  methodology for deriving the first prototype can be found in 
  \cref{sec:prototyping}. \Cref{sec:implementation} contains architecture
  details of the implementation. \Cref{sec:evaluation} presents the 
  testing methodology, the results of which can be found in \Cref{sec:results}.
  Conclusions are made in \cref{sec:conclusion}.
\end{document}
