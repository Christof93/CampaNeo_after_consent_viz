\begin{document}
  Since it has come into effect in 2018, the general data protection regulation (GDPR) has had a big impact on the way companies deal with user data of all kinds.
  The GDPR brought awareness of data ownership and protection of privacy to a new level for companies as well as consumers.
  There is a big interest to handle data generated by activities of real people in a safe and consensual way. For many companies adhering to the regulations is not only important to prevent costly legal affairs but also to keep up a good reputation with their customers.
  Customers are more and more informed about their rights as data owners. Many value their privacy high and want to control their digital footprint themselves.
  
  The GDPR requires \textbf{consent} between a data owner and a data controller if the data controller wants to process any kind of information related to the data owner.(Art. 6, 1a) This consent has to be freely given, specific, informed and unambiguous. (rec 32)
  
  The work of this paper is focused on establishing a truly informed consent, meaning that a data owner is completely aware of extent, target and content of his data sharing activities. 
  To achieve this we look into general ways to improve transparency with the data owner through informative visualisations. We believe that clever visualisations help users to get an overview of their data sharing situation more quickly and less mentally tiresome than through endless paragraphs of text.
  
  When it comes to data visualisation for an end-user, the most important aspect is the simplification of complex data. Data for applications is generally stored in some sort of database system, opposing a more complex schema, which is very complicated and hard to understand for non-experts unfamiliar with the domain. Therefore, a step of transformation has to be used to represent selected data to a user in a readable manner. Visualisation is a technique of mapping such complex data to graphical elements, thus making it easier to understand relations within the dataset. This might be straight-forward to implement for simple numerical data, as those can often be put into a simple chart or graph.

  When it comes to semantic data, the complexity is increased, since data from a \textit{triplestore} mainly describes objects and their relations. Although one could just display the whole database as a graph, with all objects and their relations, this would result in a very large visualisation, as productive \textit{triplestores} contain a very high amount of relations.

  The task of visualising data stored as triples is getting more prominent with the increase in the use of semantic technology in the web. This opens up opportunities to enrich graphical user analysis with visualisation of semantic data.
  
  % TODO: cite CampaNeo
  This paper presents an approach to visualise data in the automotive domain of the \textit{CampaNeo} project. The target is a small to medium display, typically found on larger smartphones or tablets, that displays the flow of data from a users car to third party companies. The idea of \textit{CampaNeo} is that users can share their data with other institutions like governmental agencies, universities or data processing companies who collect high amounts of data with the intent to solve problems around mobility and transport. The solutions generated from the data should in turn benefit the data owner. The visualisation focuses on highlighting those data streams to the user.
  
  In the following section we show two state of the art approaches for similar visualisation interfaces. We will point at their shortcomings and define our research question to in order to find the visualisation that solves our specific problem. The fifth section presents the initial prototype and the mock-ups. Then, we describe the architecture we used to build our first implementation. In the evaluation section, we explain how we tested our research question with a user-study and present the results of it. The last section presents a small conclusion and an outlook to future work.
\end{document}