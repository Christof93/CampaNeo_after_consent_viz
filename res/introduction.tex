\begin{document}
  When it comes to data visualisation for an end-user, the most important aspect is the simplification of complex data. Data for applications is generally stored in some sort database system, opposing a more complex schema, which is very complicated and hard to understand for non-experts unfamiliar with the domain. Therefore, a step of transformation has to be used to represent selected data to a user in a readable manner. Visualisation is a technique of mapping such complex data to graphical elements, thus making it easier to relate relations within the dataset. This this might be straight-forward to implement for simple numerical data, as those can often be put into a simple chart or graph.

  When it comes to semantic data, the complexity is increased, since data from a \textit{triplestore} mainly describes objects and their relations. Although one could just display the whole database as a graph, with all objects and their relations, this would result in a very large visualisation, as productive \textit{triplestores} contain a very high amount of relations.

  The task of visualising data stored as triples is getting more prominent with the increase in the use of semantic technology in the web. This opens up opportunities to enrich graphical user analysis with visualisation of semantic data.
  
  % TODO: cite CampaNeo
  This paper presents an approach to visualise data in the automotive domain of the \textit{CampaNeo} project. The target is a small to medium display, typically found on larger smartphones or tablets, that display the flow of data from a users car to third party companies. The idea of the project is that users intendedly share their data with other companies, who collect high amounts of data with the intent to improve the experience for users in the future. The visualisation focuses on highlighting those data streams to the user.
  
  The following section presents the initial prototype and the mock-ups including a small intermediate analysis. The third section presents the architecture that was used to implement the final system. Afterwards, a user study was conducted to evaluate the usability of the visualisation. The last section presents a small conclusion and an outlook to future work.
\end{document}