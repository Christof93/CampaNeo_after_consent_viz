\documentclass[notitlepage,11pt]{article}
\linespread{1.1}

\raggedbottom

\usepackage{float}

\usepackage{lmodern}
\usepackage[english]{babel}

\usepackage{fontspec}
\defaultfontfeatures{Ligatures=TeX}

\usepackage{multicol}

\usepackage{listings}

\usepackage{graphicx}
\graphicspath{{assets/}}

\PassOptionsToPackage{hyphens}{url}
\usepackage{hyperref}
\usepackage{cleveref}
\usepackage{nameref}
\usepackage{subfiles}

\addto\extrasenglish{
  \def\sectionautorefname{Section}
}

\usepackage[
  style   = numeric,
  sorting = none,
]{biblatex}
\bibliography{paper}

\begin{document}
\title{Consent Visualisation}
\author{Christof Bless, Lukas Dötlinger, Michael Kaltschmid \& Markus Reiter}
\maketitle
% TODO rewrite abstract
\begin{abstract}
  Data privacy scandals in the recent past have left many people around the world
  insecure about the implications of data sharing in every day life.  At the
  same time advances in machine learning and data science have given us the
  possibility to enhance the safety and the quality of life of many people.  To
  collect data from individuals nowadays companies have to rely on mutual
  consent and trust. New regulations such as the GDPR make sure that consumers
  are not exploited or taken advantage of.  From a legal standpoint it is
  necessary to inform customers to a full extent about what information is
  collected from them.  In this paper we present a new visualisation approach
  to keep people informed about the activities linked to their data sharing
  agreements. We introduce a user-centred application with a transparent
  visualisation aiming to give users a better understanding of the data sharing
  processes in the background of their consent agreements.  (Does this lead to
  more legal awareness and trust? evaluation results summary)
\end{abstract}

\section{Introduction}
\label{sec:introduction}
\subfile{res/introduction}

\section{Related Work}
\label{sec:related_work}
\subfile{res/related}

\section{Research Question}
\label{sec:research_question}
\subfile{res/research_question}

\section{Prototype}
\label{sec:prototyping}
\subfile{res/prototyping}

\section{Implementation}
\label{sec:implementation}
\subfile{res/architecture}

\section{Evaluation}
\label{sec:evaluation}
\subfile{res/results}

\section{Conclusion}
\label{sec:conclusion}
\subfile{res/conclusion}

\newpage
\printbibliography
\end{document}
