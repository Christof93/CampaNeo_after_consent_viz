\documentclass{llncs}
\linespread{1.1}

\raggedbottom

\usepackage{float}

\usepackage{lmodern}
\usepackage[english]{babel}

%\usepackage{fontspec}
%\defaultfontfeatures{Ligatures=TeX}

\usepackage{multicol}

\usepackage{listings}

\usepackage{graphicx}
\graphicspath{{assets/}}

\PassOptionsToPackage{hyphens}{url}
\usepackage{hyperref}
\usepackage{caption}
\usepackage[nameinlink, noabbrev]{cleveref}
\usepackage{nameref}
\usepackage{subfiles}

\addto\extrasenglish{
  \def\sectionautorefname{Section}
}


\title{Raising Legal Awareness Through User-Centred Consent Visualisation}
\author{Christof Bless \and Lukas Dötlinger \and Michael Kaltschmid \and Markus Reiter \and Anelia Kurteva \and Antonio J. Roa-Valverde \and Anna Fensel }
\institute{Institute of Computer Science, University of Innsbruck}


\begin{document}

\maketitle

\begin{abstract}
  Semantic knowledge graphs facilitate systematic large-scale data analysis by providing machine readable structures which can be shared across different platforms.
  Nowadays, knowledge graphs are used to standardise collection and sharing of user data in many different sectors. 
  %Advances in machine learning and data science have given us the
  %possibility to enhance the safety and the quality of life of many people through large scale data analysis. To
  %collect data from individuals nowadays companies have to rely on mutual
  %consent and trust. At the
  %same time data privacy scandals in the recent past have left many people around the world
  %insecure about the implications of data sharing in everyday life.
  Regulations such as the GDPR make sure that consumers
  are not exploited or taken advantage of when they share their user data.
  From a legal standpoint it is
  necessary to inform customers to a full extent about what information is
  collected from them. In this paper we present a new visualisation approach
  to keep people informed about the activities linked to their data sharing
  agreements. We introduce a user-centred application with a transparent knowledge graph
  visualisation aiming to give users a better understanding of the data sharing
  processes in the background of their consent agreements. Finally, we show
  the results of a user study conducted to find out whether this
  visualisation leads to more legal awareness and trust.
  We show that our approach can increase data sharing consent rates by over 20\% in different types of use cases.
\end{abstract}

\keywords{Knowledge Graph Visualisation \and GDPR \and consent \and legal awareness}

\section{Introduction}
\label{sec:introduction}
\subfile{res/introduction}

\section{Related Work}
\label{sec:related_work}
\subfile{res/related}

\section{Personal data consent}
\label{sec:research_question}
\subfile{res/research_question}

\section{Implementation}
\label{sec:implementation}
\subfile{res/implementation}

\section{Evaluation}
\label{sec:evaluation}
\subfile{res/evaluation}

\section{Results}
\label{sec:results}
\subfile{res/results}

\section{Future Work}

\section{Conclusion}
\label{sec:conclusion}
\subfile{res/conclusion}

\section{Acknowledgements}
\label{sec:acknoledgements}
\subfile{res/acknowledgements}

\bibliographystyle{splncs04}
\bibliography{paper}

\end{document}
